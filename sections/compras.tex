\section{13. Gestión de compras}
\label{sec:compras}

La gestión de compras comprenderá dos partes fundamentales:
\begin{itemize}
\item Compras a proveedores externos.

Se optará por realizar la compra de las placas de circuito impreso de los nodos a un proveedor externo ya que presenta costos realmente bajos con capacidad de entrega del producto en 7 días. La compra se resolverá completamente online.
Dado que en la Municipalidad de Rosario los trámites necesarios para llevar a cabo una gestión de compras en el exterior son complejos y harían que el proyecto sufra demoras muy importantes, se prevé realizar una reunión con las autoridades tanto políticas como administrativas para lograr una agilización en esta gestión de compra. En dicha reunión se pondrá de manifiesto la importancia del proyecto en el contexto que se avecina referido al almacenamiento seguro de especialidades medicinales.
Se estima, en el contexto mencionado, la posible recepción de importantes partidas de vacunas para el abordaje de un plan masivo de vacunación, durante el cual resultará crítico el control de su almacenamiento.
Dada esta situación, estimamos que será prioridad para las autoridades de la Secretaría de Salud Pública, agilizar las gestiones de compra, para lograr la fluidez requerida para la consecución de los objetivos del proyecto.

\item Compras a proveedores locales.\\
Se aplicarán las normativas de compras de la Municipalidad de Rosario para adquirir los componentes electrónicos en proveedores locales. Se iniciarán los trámites administrativos según el monto del gasto a realizar.
\begin{itemize}
\item Gastos menores a \$20.000 se harán por compra directa. El criterio adoptado será adjudicar la compra del producto especificado al proveedor que ofrezca el menor costo.
\item Gastos mayores a \$20.000 se harán por licitación pública, estando a cargo de la confección de los pliegos el responsable del proyecto. 
\end{itemize}
\end{itemize}

Statement of Work\\
Se prevé que sólo sea utilizada la modalidad compra directa, ya que se espera que ninguna supere el valor de \$ 20.000.-

En la figura \ref{nota_pedido} se muestra la nota de pedido para la compra de los componentes electrónicos.

En la figura \ref{compra_directa} se muestra el pliego confeccionado para la modalidad compra directa.