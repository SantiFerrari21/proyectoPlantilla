\section{7. Matriz de uso de recursos de materiales}
\label{sec:recursos}
En la tabla \ref{recursos1}, se visualiza la utilización de los recursos. Las cantidades están expresadas en horas.

Descripción de los recursos necesarios.

PC desktop: será una computadora de escritorio con sistema operativo Ubuntu 20.04 con las utilidades de oficina instaladas.
\begin{itemize}
\item Texmaker
\item LibreOffice.
\item Chrome con extensión Gantter.
\item Git.
\end{itemize}

Laboratorio: estará compuesto por herramientas e instrumental para desarrollo y armado de placas electrónicas.
\begin{itemize}
\item Osciloscopio.
\item Estación de soldadura/desoldadura.
\item Generador de ondas arbitrarias.
\item Fuente de alimentación.
\item Multímetro de banco.
\item PC de desarrollo con las utilidades requeridas, Visual Studio Code, Matlab, Git.
\item Módulos nodeMCU con microcontroladores ESP8266.
\item Red WiFi con permisos de administración.
\end{itemize}

Servidor: un hardware de servidor donde se instalará la dashboard. Deberá poseer al menos 4 Gb de memoria RAM y capacidad de almacenamiento en disco rígido de al menos 30 Gb. 

Sala de reuniones: espacio de trabajo para albergar al menos un grupo de 10 personas, deberá poseer pantalla y computadora con conexión a Internet.

Teléfono móvil: un teléfono propio para hacer las pruebas de recepción de alarmas.

\begin{table}

\centering
\resizebox{\textwidth}{!}{
\begin{tabular}{|c|c|c|c|c|c|c|}
%\begin{tabularx}{\linewidth}{@{}|c|X|X|X|X|X|c|@{}}
\hline
\cellcolor[HTML]{C0C0C0} & \cellcolor[HTML]{C0C0C0} & \multicolumn{5}{c|}{\cellcolor[HTML]{C0C0C0}Recursos requeridos (horas)} \\ \cline{3-7} 
\multirow{-2}{*}{\cellcolor[HTML]{C0C0C0}\begin{tabular}[c]{@{}c@{}}Código\\ WBS\end{tabular}} & \multirow{-2}{*}{\cellcolor[HTML]{C0C0C0}\begin{tabular}[c]{@{}c@{}}Nombre \\ tarea\end{tabular}} & PC desktop & Laboratorio & Servidor & Sala reuniones & Teléfono móvil \\ \hline
 
 1.1&Definición de alcances & 9 & &  &  &  \\ \hline
 1.2&Selección de efectores &  &  &  & 3 &  \\ \hline
 1.3&Selección de usuarios  &  &  &  & 3 &\\ \hline
 1.4&Charlas con usuarios  &  &  &  & 6 &\\ \hline
 1.5&Estudio de dashboards & 10 &  &  &  &\\ \hline
 1.6&Escritura del plan de trabajo & 15 &  &  &  &\\ \hline
 2.1&Estudio de sensores  & 2 &  &  &  &\\ \hline
 2.2&Estudio y simulación filtro  &  & 6 &  &  &\\ \hline 
 2.3&Pruebas sensor filtro  &  & 15 &  &  &\\ \hline
 2.4&Investigación microcontroladores  & 4 &  &  &  &\\ \hline
 2.5&Investigación bibliotecas  & 4 &  &  &  &\\ \hline
 2.6&Desarrollo PCB  &  & 30 &  &  &\\ \hline
 2.7&Montaje PCB  &  & 8 &  &  &\\ \hline
  3.1&Estudio bibliotecas WiFi & 15 &  &  &  &\\ \hline
 3.2&Gestión de certificados TLS  & 10 &  &  &  &\\ \hline
 3.3&Desarrollo comunicación  &  & 40 &  &  &\\ \hline
 3.4&Prueba comunicación  &  & 40 &  &  &\\ \hline
 3.5&Procesamiento variable  &  & 15 &  &  &\\ \hline
 3.6&Desarrollo página configuración &  & 35 &  &  &\\ \hline
 3.7&Prueba del conjunto  &  & 40 &  &  &\\ \hline
 3.8&Depuración de errores  &  & 40 &  &  &\\ \hline
 4.1&Instalación SO servidor  &  &  & 8 &  &\\ \hline
 4.2&Gestión de usuarios y permisos &  &  & 8 &  &\\ \hline
 5.1&Instalación dashboard  & 4 &  & 4 &  &\\ \hline 
 5.2&Aprendizaje uso dashboard  & 15 &  & 10 &  &\\ \hline
 5.3&Gestión usuarios dashboard  & 5 &  & 5 &  &\\ \hline
 5.4&Creación de paneles  & 20 &  & 20 &  &\\ \hline
 5.5&Prueba del conjunto  & 20 &  & 20 &  &\\ \hline
  6.1&Creación canales Telegram  & 2 &  &  &  &\\ \hline
 6.2&Instalación app Telegram &  &  &  &  &2\\ \hline
 6.3&Creación reglas de alarmas & 10 &  & 10 &  &\\ \hline
 6.4&Creación reglas de estado &  10 &  & 10 &  &\\ \hline
 6.5&Prueba alarma temperatura & 15 &  &  &  &5\\ \hline
 6.6&Prueba alarma estado & 15 &  &  &  &5\\ \hline
 7.1&Verificación de requerimientos & 20  &  &5  &  &5\\ \hline
 8.1&Escritura de manuales & 16 &  &  &  &\\ \hline
 8.2&Escritura de memoria final & 40 &  &  &  &\\ \hline
 8.3&Elaboración de la presentación & 20 &  &  &  &\\ \hline
 \end{tabular}%
 }
 \caption{\textit{Tabla de asignación de recursos}}
 \label{recursos1}
 \end{table}
