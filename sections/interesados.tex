\section{Identificación y análisis de los interesados}
\label{sec:interesados}

 
%Nota: (borrar esto y todas las consignas en color rojo antes de entregar este documento).
 
%Es inusual que una misma persona esté en más de un rol, incluso en proyectos chicos.
 
%Si se considera que una persona cumple dos o más roles, entonces sólo dejarla en el rol más importante. Por ejemplo:

%\begin{itemize}
%\item Si una persona es Cliente pero también colabora u orienta, dejarla solo como Cliente.
%\item Si una persona es el Responsable, no debe ser colocado también como Miembro del equipo.
%\end{itemize}

%Pero en cambio sí es usual que el Cliente y el Auspiciante sean el mismo, por ejemplo.

\begin{table}[ht]
%\caption{Identificación de los interesados}
%\label{tab:interesados}
\begin{tabularx}{\linewidth}{@{}|l|X|X|l|@{}}
\hline
\rowcolor[HTML]{C0C0C0} 
Rol           & Nombre y Apellido & Organización 	& Puesto 	\\ \hline
%Auspiciante   &                   &              	&    -    	\\ \hline
Cliente       & \clientename      &\empclientename	&      Director de Bioingeniería\\ \hline
Impulsor y colaborador      & Roberto Collelo  & \empclientename	&Sub Director de Bioingeniería\\ \hline
Responsable   & \authorname       & FIUBA        	& Alumno 	\\ \hline
Colaboradores & 
                Silvana Pereyra &Dirección Centros de Salud	 &Jefa Droguería Central\\ \hline
Orientador    & \supname	      & \pertesupname 	& Director	Trabajo final \\ \hline
%Equipo        & miembro1 \newline 
%				miembro2          &       -       	&      -  	\\ \hline
%Opositores    &    -               &      -        	&     -   	\\ \hline
%Usuario final &      -             &      -        	&      -  	\\ \hline
\end{tabularx}
\end{table}


%Sería deseable listar a continuación de la tabla las principales características de cada interesado.
 
%Por ejemplo:
\begin{itemize}
%\item Auspiciante: es riguroso y exigente con la rendición de gastos. Tener mucho cuidado con esto.
%\item Equipo: Juan Perez, suele pedir licencia porque tiene un familiar con una enfermedad. Planificar considerando esto.
\item Colaboradores

Roberto Collelo: resulta de valiosa ayuda en el desarrollo del firmware de los sensores, además de promover el proyecto en las estructuras de mando de la Secretaría de Salud Pública.

Silvana Pereyra: su intervención será desde el lado usuarios, ya que se muestra como una facilitadora para la implementación del sistema.
\end{itemize}