\section{6. Desglose del trabajo en tareas}
\label{sec:wbs}


\begin{enumerate}
\item Planificación general. (46 hs)
	\begin{enumerate}
	\item Definiciones de alcances, requerimientos y presupuestos. (9 hs).
	\item Selección de efectores donde colocar los prototipos. (3 hs).
	\item Selección de usuarios que harán las pruebas del prototipo. (3 hs).
	\item Charlas previas con los usuarios seleccionados para explicar el por qué de la implementación y así facilitar su adopción. (6 hs)
	\item Estudio y selección de las dashboards disponibles. (10 hs).
	\item Escritura del plan de trabajo final. (15 hs).
	\end{enumerate}
\item Planificación y desarrollo del circuito electrónico y PCB del sensor. (69 hs).
	\begin{enumerate}
	\item Estudio y selección de sensores de temperatura. (2 hs).
	\item Estudio, cálculo y simulación del filtro activo para el sensor de temperatura. (6 hs).
	\item Desarrollo y pruebas del circuito sensor-filtro activo. (15 hs).
    \item Investigación de los microcontroladores aptos para el proyecto. (4 hs).	
	\item Investigación de las bibliotecas disponibles para el microcontrolador seleccionado. (4 hs).
	\item Desarrollo de la placa PCB del sensor. (30 hs).
	\item Montaje de los componentes en 4 placas PCB. (8 hs).
	\end{enumerate}
\item Planificación y desarrollo del firmware del sensor 235.  (hs).
	\begin{enumerate}
	\item Estudio del funcionamiento de las bibliotecas para conectividad WiFI del microcontrolador. (15 hs).
	\item Estudio y elaboración de los certificados TLS.  (10 hs).
	\item Desarrollo de las funciones de comunicación utilizando protocolo de seguridad. (40 hs).
	\item Pruebas y depuración de errores en la conexión y transporte del dato al servidor central. (40 hs).
	\item Desarrollo de las funciones de procesamiento de la variable medida. (15 hs).
	\item Desarrollo de la página web de configuración. (35 hs).
	\item Prueba del conjunto. (40 hs).
	\item Depuración de errores. (40 hs).
	\end{enumerate}
\item Instalación y configuración del sistema operativo del servidor central. (16 hs).
	\begin{enumerate}
	\item Armado de máquina virtual en ESXi e instalación del sistema operativo Linux/Debian 8.0.  (8 hs).
	\item Instalación y configuración de usuarios, permisos y servicios escenciales.  (8 hs).
	\end{enumerate}	
	
\item Instalación y configuración de la dashboard en el servidor central.  (123 hs).
	\begin{enumerate}
	\item Instalación de la dashboard y su base de datos asociada.(8 hs).
	\item Aprendizaje del uso de la dashboard.(25 hs).
	\item Creación y configuración de permisos de los usuarios a la dashboard. (10 hs).
	\item Creación de los paneles para usuarios administradores, jefes y operadores. (40 hs).
	\item Prueba y depuración de errores del conjunto. (40 hs).
	\end{enumerate}		
	
\item Gestión de las notificaciones. (84 hs).
	\begin{enumerate}
	\item Creación de canales en Telegram. (2 hs).
	\item Instalación de app Telegram en usuarios seleccionados para prueba. (2 hs).
	\item Creación de la cadena de reglas en dashboard para el envío de alarmas. (20 hs).
	\item Creación de la cadena de reglas en dashboard para mostrar el estado del dispositivo. (20 hs).
	\item Pruebas de alarmas de baja y alta temperatura. (20 hs).
	\item Pruebas de alarmas de offline y online de los dispositivos. (20 hs).
	\end{enumerate}		

\item Verificación de todas las funcionalidades. (30 hs).
	\begin{enumerate}
	\item Verificación del cumplimiento de los requerimientos.  (30 hs).
	\end{enumerate}		

\item Cierre.  (76 hs)
	\begin{enumerate}
	\item Escritura de la documentación para usuarios. (16 hs).
	\item Escritura de la memoria final. (40 hs).
	\item Elaboración de la presentación. (20 hs).
	\end{enumerate}
	
\end{enumerate}

Cantidad total de horas: 663 hs